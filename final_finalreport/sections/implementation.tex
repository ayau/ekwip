The technology behind eKwip is able to capture the movements of the knee and allow athletes and their physicians to monitor the motion of the leg. The system consists of several sub-components, each of which performs an important function as part of the system as a whole. The sub-components include a spandex wrap, a microcontroller, sensors, a wireless module, and a webserver.

\subsection {Wireless Module}
In order to allow coaches to monitor the performance of athletes on the field and doctors to monitor the recovery of injured patients, a wireless interface that allows eKwip measurements to be displayed in real time was implemented. In order to achieve wireless communication, a Wifly module "!!! PLEASE CITE" was integrated into the system. The Wifly is a small Wifi board that communicates via UART serial. Having a wireless connection is useful because data can be streamed in real time to the server that is running in order to provide an intuitive visualization to users and/or their doctors. The only constraint with this is that the user must be in range of a wireless network.

The Wifly is configured to send a packet each time a data point is received. This, along with an increased baud rate on the serial line between the mBed and the Wifly module, allowed for a wifi sampling rate of up to 100 Hz.

\subsection {Server}
The server is implemented with Node.js "!!! PLEASE CITE" and communicates with the eKwip wrap via websockets "!!! PLEASE CITE". Websockets were chosen in order to provide a stream of information from the wrap to the server. The server collects the data streamed from the wrap and in turns communicates to the front-end interface. "!!! EXPLAIN WHAT LOGIC" Some simple logic is implemented on the server to normalize and calibrate the data sent from the wrap, as well as filter out possible measurement errors. The front-end currently displays a 3D model of the knee given the data from eKwip with WebGL "!!!CITE". However, in the future, the front-end will also allow multiple users to log in to view indivisualized reports and statistics, and share this information with their physician. This can allow patients and doctors to monitor the performance of the knee as well as observe the recovery of a patient over time.
