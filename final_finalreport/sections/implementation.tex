The technology behind eKwip is able to capture the movements of the knee and will accurately create an industry standard Knee Injury Risk Index, or KIRI for short. The system consists of several sub-components, each of which performs an important function as part of the system as a whole. The sub-components include a neoprene wrap, a microcontroller, sensors, a wireless module, a Micro SD card, and a webserver.

\subsection {Wireless Module}
In order to allow coaches to monitor the performance of athletes on the field and doctors to monitor the recovery of injured patients, a wireless interface that allows eKwip measurements to be displayed in real time was implemented. In order to achieve wireless communication, a Wifly module "!!! PLEASE CITE" was integrated into the system. The Wifly is a small Wifi board that communicates via UART serial. Having a wireless connection is useful because data can be streamed in real time to the server that is running in order to provide an intuitive visualization to users and/or their doctors. The only constraint with this is that the user must be in range of a wireless network.

"Initially, the wifi module was streaming the data extremely slowly. It was determined that this was due to packetization of the data and a very long data string being sent with each data point."" In order to mitigate this problem, the data string was shortened, and the Wifly was configured to send a packet each time a data point is received. This improved the wifi sampling rate to 10 Hz, which will be increased by raising the baud rate on the serial line between the mBed and the Wifly.

\subsection {Micro SD Card}
If no network is present, eKwip will log all the data that is being received to a Micro SD card so that the wearer can access their data in a later point. The Micro SD card reader will be attached to the mBed via UART serial. It will be written using a simple file format that will allow parsing of the data when in range of a wireless network. Alternatively, the Micro SD card can be removed and connected to a computer in order to import the data.

\subsection {Server}
The server is implemented with Node.js "!!! PLEASE CITE" and communicates with the eKwip wrap via websockets "!!! PLEASE CITE". Websockets is chosen in order to provide a stream of information from the wrap to the server. The server collects the data streamed from the wrap and in turns communicates to the front-end interface. "!!! EXPLAIN WHAT LOGIC" Some simple logic is implemented on the server to normalize and calibrate the data sent from the wrap, as well as filter out possible measurement errors. The front-end currently displays a 3D model of the knee given the data from eKwip with webgl "!!!CITE". However, in the future, the front-end will also include additional information, show forces and acceleration on the knee model, as well as a graph showing the risk of ACL injury over the time that it was measured. This can allow patients and doctors to monitor the performance of the knee as well as observe the recovery of a patient overtime.
