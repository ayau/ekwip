At its current point, eKwip is in the prototype stage. While the current implementation is able to collect data on the wearer's knee movements as well as transmit that data to the server, there is no conclusive metric being calculated for the prototype nor is there a way for the information to be displayed. 

\subsection {Knee Injury Risk Index}
The first step moving forward is to come up with an algorithm to calculate the Knee Injury Risk Index (KIRI). In order to do this, certain factors, such as the calculated Q-angle or knee adduction moment during movement, after being calculated, will be given certain weights and a final KIRI value will be returned for the wearer "!!! HOW". The Q-angle is formed from a line drawn from the Anteriro Superior Iliac Spine (the front of the pelvic bone at the hip level) to the center of the kneecap, and from the center of the kneecap to the Tibial Tubercle (a bump about 5 centimeter below the kneecap on the front of the Tibia "!!! conley2007female MISSING CITATION". The knee adduction moment determines how force is distributed at the knees "!!! noyes2010knee MISSING CITATION". A higher Q-angle means that the wearer has an increased risk of ACL injuries, while the knee adduction moment of a person in movement will affect how certain loads are applies to their knees and ACL. 

\subsection {Machine Learning}
Another important issue that needs to be addressed is the variability of gaits, resting positions, and Q-angles from person-to-person. Designing a wrap that fits everyone will not generate accurate results as various athletes wear the wrap. The proposed solution to this issue is to produce a basic machine learning algorithm that will cause eKwip to adapt to wearer. The algorithm will likely collect data on the wearer initially. Then, using this information, eKwip will adjust its collection formulas to fit with how the wearer normally rests or moves. This aspect of the project will likely be the most difficult; therefore, a major portion of the remaing time will be spent on creating and validation the implementation. "!!! EXPLAIN HOW WE ARE GETTING DATA"

\subsection {Graphical User Interface}
Presenting the data and information collected by eKwip will be useful for physical therapists or coaches. For situations when an atheletes starts to complain about either soreness and recieve a high KIRI score, physical therapists or coaches should be able to go to the site and see a timeline or graph of the wearer's various data points over time. This will give the physical therapist or coach a better idea of how their athlete is moving about and should give a notion of what needs to be fixed. In addition, as the server currently shows how the athlete's knee is moving about in real-time, a future implementation of this feature could include a replay of the movements about five or ten minutes prior.

\subsection {Testing \& Validation}
Testing and validation of all data collection and the system itself will be carried out along with the implementation of the previously mentioned features. Validation is critical for the system and will determine whether or eKwip is actually useful as an aid for physical therapists or coaches. Fortunately, access to equipment at the Penn Sports Medicine Center will allow us to test eKwip against what is currently being used. Because of the nature and focus of the project, certain data will be very risky or even dangerous to collect on an actual person. Fortunately, knee models at the Penn Sports Medicine Center will allow us to collect this information as well as give us methods to gather data on the knee in various positions or stances. Moving forward, testing and validation will be continually done, with majors tests performed when major milestones are completed on the project.

\subsection {Reach Goals}
Based on how fast these features can be implemented and how much time remains, the reach goal set for the project may be pursued: to implement a prevention measure for eKwip that will be an extension of the prediction. However, this largely depends on how fast reads and calculations on the system itself can be performed as well as how quickly accuracy of the values can be achieved. Prevention also requires major research into potential materials to use, as eKwip would need a method to brace the sudden movement of the wearer to lessen or prevent the injury. In addition, a Prevention implementation will modify the rate and method of data collection and calculation.
