At its current point, eKwip is in the prototype stage. While the current implementation is able to collect data on the wearer's knee movements as well as transmit that data to the server, there is no conclusive metric being calculated for the prototype. As such, the first step moving forward is to come up with an algorithm to calculate the Knee Injury Risk Index (KIRI). In order to do this, certain factors, such as the calculated Q-angle or knee adduction moment during movement, after being calculated, will be given certain weights and a final KIRI value will be returned for the wearer. The Q-angle is formed from a line drawn from the Anteriro Superior Iliac Spine (the front of the pelvic bone at the hip level) to the center of the kneecap, and from the center of the kneecap to the Tibial Tubercle (a bump about 5 centimeter below the kneecap on the front of the Tibia (TODO Add a citation for this). The knee adduction moment determines how force is distributed at the knees (TODO Add citation). A higher Q-angle means that the wearer has an increased risk of ACL injuries, while the knee adduction moment of a person in movement will affect how certain loads are applies to their knees and ACL. 

Another important issue that needs to be addressed is the variability of gaits, resting positions, and Q-angles from person-to-person. Designing a wrap that fits everyone will not generate accurate results as we move between athletes. Our solution to this issue is to produce a basic machine learning algorithm that will cause eKwip to adapt to wearer. The algorithm will likely collect data on the wearer initially. Then, using this information, eKwip will adjust its collection formulas to fit with how the wearer normally rests or moves. This aspect of the project will likely be the most difficult; therefore, we will be spending a major portion of the remaing time on creating and validation the implementation.

Testing and validation of all data collection and the system itself will be carried out along with the implementation of the previously mentioned features. Validation is critical for the system and will make-or-break whether or eKwip is actually useful as an aid for physical therapists or coaches. Fortunately, access to equipment at the Penn Sports Medicine Center will allow us to test eKwip against what is currently being used. Because of the nature and focus of our project, certain data will be very risky or even dangerous to collect on an actual person. Fortunately, knee models at the Penn Sports Medicine Center will allow us to collect this information as well as give us methods to gather data on the knee in various positions or stances. Moving forward, testing and validation will be continually done, with majors tests performed when major milestones are completed on the project.

Based on how fast these features can be implemented and how much time remains, we may pursue the reach goal: to implement a prevention measure for eKwip that will be an extension of the prediction. However, this largely depends on how fast we can perform reads and calculations on the system itself as well as how quickly we can ensure the accuracy of the values. Prevention also requires major research into potential materials to use, as eKwip would need a method to brace the sudden movement of the wearer to lessen or prevent the injury. In addition, a Prevention implementation will modify the rate and method of data collection and calculation. It is likely we may not reach this stage of the project, but, as stated, this is all highly dependent on the time constraints. 