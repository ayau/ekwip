Related work for this project is divided among three fields: algorithms for ACL injury detection (Section 2.1), knee brace effectiveness/performance hindrance (Section 2.2), and Smart Materials (Section 2.3). ACL injury detection covers previous research in minimizing the testing required to detect whether an athlete is at risk of suffering an ACL injury. Knee brace effectiveness/performance hindrance deals mainly with studies of the extent to which a Functional Knee Brace and a Prophylactic Knee Brace play in the ability for an athlete to perform on the field. Finally, Smart Materials concern the reach for this project and the possibility of actually preventing the ACL injuries from occurring on the field.

\subsection{ACL Injury Detection} Prior research in ACL Injury Detection have shown that techniques exist that accurately capture and analyze various measures relating to the knee to determine the probability of ACL injuries \cite{smedicine}. Using such metrics as knee valgus motion, knee flexion rotation of motion, and body mass, researchers were able to come up with a way to determine knee abduction moments (KAM), which are used to identify whether or not an athlete is at high risk for an ACL injury, with a sensitivity of 77\% and specificity of 71\% \cite{smedicine}\cite{Bahr01062005}. 

\subsection{Knee Brace Effectiveness/ Performance Hindrance} Prior research in using Functional or Prophylactic knee braces out in the field useful but results on the potential hindrance of using these kinds of braces were inconclusive \cite{Myer01042011}. Knee braces, especially Functional Knee Braces (FKB), which are more mechanical in nature and thus more obtrusive, are shown to provide “20-30\% greater knee ligament protection”. This suggests that FKBs have an impact in reducing the severity of knee injuries. More testing needs to be done to see whether or not FKBs will actually hinder the performance of an athlete. Another important factor of the effectiveness of knee braces is in rehabilitation, where a combination of exercises and brace use can speed up recovery \cite{hewett2010acl} .

\subsection{Smart Materials} Prior research in the use of Electro-rheological fluids (ERF) show that these material reacts quickly when presented with an electric field, have a high yield stress, and are very lightweight and easily molded \cite{smaterials}. ERF can quickly change viscosities, which make it an excellent material to use in the project. Due to the very nature of ERF’s fast response and simple interface, using it in a light, functional knee brace would allow lead to an easy implementation of the preventative nature of the project.