The project is split into three sections - measurement, prediction and prevention. The project will focus on achieving all the goals in measurement and prediction, while attempting to tackle prevention if time permits.

\subsubsection{Phase 1 - Measurement}
This part of the project focuses on the research and implementation of the knee wrap to collect measurements and monitor movements in the knee. The wrap should be able to store the collected data and relay it through a user friendly interface. At the end of this phase, a basic functioning prototype should be created.

The first part of this phase includes the research and understanding of ACL injuries in order to decide which measurements to take. In order to better understand the factors that contribute to an ACL injury - the different angles, forces and orientation of the knee, doctors and experts in field will be consulted. The secondary purpose of meeting with experts in the field is to identify the group of people with the highest risk of injury, which provides a better focus for the project. As it currently stands, it seems that female athletes are at higher risk of sporting ACL injuries - a rate of 2 to 10 times more likely to sustain ACL injuries compared to male athletes playing the same landing and cutting sports \cite{smaterials}. The potential value of this knee wrap can also be estimated by determining the benefits to different fiends in the current market, such as rehabilitation and sports health.

Once the necessary measurements needed to predict an injury is determined, a prototype of the wrap will be built. The basic prototype will consist of a simple elastic knee wrap, an mbed microcontroller (LPC1768) and sensors (Pololu UM6-LT Orientation Sensor). The microcontroller will be programmed and connected to the sensors and the sensors will be calibrated to give precise measurements. The prototype wrap will be worn and tested and measurements collected. The range of the different measurements according to the constraints of the knee will be observed which will allow further calibration of the sensors. Different sampling rates will be tested and the one with the highest precision to memory cost ratio will be used.

The prototype wrap should also be able to store the measurements in its internal memory and able to output to a computer via USB or ethernet connection. A simple user interface will be built in order to give some sort of visualization of the knee movements over the time frame as well as the different measurements collected.

\subsubsection{Phase 2 - Prediction}
This part of the project focuses mainly on the prediction aspect of eKwip. The goals of this phase includes implementation of an algorithm to determine the risk of ACL injury given the measurements. A risk index will be compiled to indicate the knee performance of wearer and the risk of injury. Since different people have different ranges of tolerable movements, it is necessary for the wrap to be flexible and adaptive, in order to determine the risk of injury for each individual. Secondly, the prototype at the end of this stage should be able to predict the risk of injury in real time, with precision up to tens of milliseconds. This allows possible real time prevention of the injury once the prevention mechanism is developed in phase 3. Lastly, the knee wrap should be able to communicate with the server wirelessly to transmit data of his or her movements to allow the wearer to be monitored in real time.

The research portion of this phase includes collection of data, determining the risk of injury given a set of movements and creating a risk index. This requires further meetings with experts in the field to better understand the biomechanics of the knee and limits of the ACL given a set of measurements, such as position, orientation, speed and load. Since testing on human subjects is not a possibility, research in this phase is very important to determine the risk of injury and to come up with the risk index.

Once the risk index has been determined, the functionality of the prototype knee wrap can then be extended to include real time prediction of injury. This is done by predicting the future position, orientation and load on the knee based on current measurements. In order to achieve real time prediction, the sampling rate of the measurements will be increased in order to give a more precise reading. However, due to the memory limitations, the measurements from this increased sampling rate will not be stored to memory. Furthermore, a learning algorithm will be implemented in order for the knee wrap to adjust to the different range of movements of each individual in order to provide a personalized risk index and injury prediction. Based on the complexity and processing power required for this learning algorithm, the algorithm will either reside on the microcontroller or on the server side, syncing only the necessary information to and from the knee wrap when it is connected. With the latter implementation, the server should have a database to keep track of the data for each wearer. Lastly, a wireless module (WIFI or Bluetooth) will be added to eKwip to allow it to wirelessly connect to the server whenever it is in range. This reduces the trouble of having to physically remove and connect the wrap to a computer via a cord.

\subsubsection{Phase 3 - Prevention}
The last part of the project, focusing on injury prevention, will be completed if time permits. In this phase, a prevention mechanism will be designed and implemented into eKwip. The goal of the prevention mechanism is to reduce the impact of injury and possibly preventing the injury completely.

The prevention mechanism needs to be unobtrusive when not engaged to allow full movements of the knee, while preventing the knee from bending past at certain angles when engaged. A possible approach is the use of smart materials, in particular electrorheological fluids and magnetorheological fluids. These material will act as a fluid normally, while changing its viscosity and transforming into a solid state when either an electric field or magnetic field is applied. If these materials can be integrated into eKwip, it would be placed on the outside of the leg parallel to the knee to lessen the impact of the valgus force, which often contributes to an ACL injury, and placed vertically along the kneecap to directly protect the ACL. Once eKwip is able to predict when an injury is imminent, it can trigger an electric or magnetic field to change the viscosity of the smart materials in stages to absorb the impact and hopefully prevent the injury. Areas of further research involves discovering other possible approaches to eKwip’s prevention mechanism, analyzing the side effects of such prevention mechanisms, and testing their effectiveness. 

