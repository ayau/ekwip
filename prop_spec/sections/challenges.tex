There are several technical and non-technical challenges expected from this project. The first major challenge includes determining the variables to be measured from the knee in order to effectively predict the risk of injury. This includes a lot of experimenting, researching and consulting experts in field to correctly identify the features that best indicates an imminent ACL injury. Since there is a trade off between measuring and collecting as much information as possible, and the storage cost and processing power of the microcontroller, only the critical variables should be identified and used in the prediction process. Given the set of necessary measurements, such as relative angle, speed and orientation of knee, the placement of the sensors need to be designed to be as unobtrusive to the wearer as possible while still able to collect precise and accurate measurements from the knee. With this set of measurements defined, the knee can be effectively modelled. Currently, aside from the knee valgus motion, knee flexion range of motion, and body mass, there may be more measurable factors directly contributing to ACL injury that have yet to be determined, and thus will be a huge challenge for this project \cite{Myer01042011}.

The second major challenge involves determining the risk of injury for each individual. Since the different wearers are likely going to have different range of tolerable movements based on their fitness, gender, age and training, it is necessary for the wrap to be adaptive and flexible. Thus, the wrap needs to be able to use the collected data to make adjustments to its prediction algorithm through reinforcement learning. The challenge, however, is that the knee wrap will mostly be exposed to healthy movements in which an injury is not about to occur. It lacks the data for when an injury is occurring and therefore negative feedback is not possible. Therefore, the learning algorithm needs to be designed to be able to learn the range of movements of the wearer with only the limited dataset available.

Another major challenge is to come up with a risk index given the collection of data. Since the goal of the risk index is for it to eventually become a universal way to estimate the risk of ACL injury, it is necessary for the index to be as accurate as possible, and customized to the physique of the wearer. The construction of such an index is very complicated as it needs to take into account of the physique of the individual wearing the wrap, combined with the index currently used in rehabilitation, and assessments used to judge whether athletes are ready to return to sports after an injury \cite{Sinkjar1991209}. Although this is an ambitious task, and haven’t been accomplished before, coming up with a universal risk index to assess the possibility of ACL injury and reinjury is very useful for monitoring the performance of athletes or injured patients.

A major technical challenge to this project is to be able to collect measurements at a high enough frequency to predict and prevent the ACL injury in real time. Depending on the frequency of measurements needed to allow enough time for the microcontroller to calculate and react, some parts of the code may have to be written in a low level language, such as assembly, to decrease the processing time. The first challenge here is to identify these bottlenecks in the code and the subsequent challenge is to optimize those sections to run as fast as possible without sacrificing precision and accuracy of the injury prediction. This can allow eKwip to detect and predict ACL injuries in real time and react as fast as tens of milliseconds.

The project also presents some design challenges. The eKwip wrap should be designed as comfortable and unobtrusive as possible to not hinder the movements of athletes and patients until an injury is imminent. The wrap should be light and flexible to encourage athletes to wear it even if they have a healthy knee. This requires careful choice of microcontrollers, sensors and other components in the wrap, as well as the positioning of such components. Ultimately, eKwip should be designed to be flexible to fit athletes and patients of different body types.

Lastly, the project also presents some material and mechanical challenges. In order to prevent the ACL injury, the wrap has to be equipped with a prevention mechanism capable of absorbing huge forces and supporting the weight of the wearer. A possible approach, as mentioned in 3.1.3, is the use of smart materials that is capable of changing viscosity. However, in order to create substantial support for the knee, a large electric field or magnetic field has to be used. Depending on the size of field, this approach may be impossible for a portable wrap like eKwip. This will be explored further if time permits. 