In order to determine whether eKwip performs as intended, it is necessary to evaluate the knee wrap after the prototype is built and compare its performance with the current knee wraps. Since there are currently no knee wraps on the market that accomplishes all of what eKwip sets out to do, eKwip's different functionalities are to be compared with the different equivalent products and services out there.

To determine the effectiveness and accuracy of the injury prediction and risk assessment part of eKwip, one can compare it to other commonly used knee stability assessments in rehabilitation centers , such as the use of electromyographic measurements from muscles acting on the knee \cite{Sinkjar1991209}. Patients who suffered ACL injury can wear the eKwip knee wrap and their performance on those knee stability assessments can be compared to the risk prediction rating generated by eKwip to see if eKwip correctly identifies the health of the injured knee. If a correlation is found, then eKwip is able to successfully and reliably measure the health of a person’s knee without needing him or her to undergo a set of assessments under supervision. 

Other than performing correlation analysis on eKwip, one can also judge the quality and effectiveness of eKwip through the opinions of experts in the field. Doctors and physical therapists can use eKwip to monitor their patients and determine whether eKwip is easier and as accurate at monitoring patients over the traditional ways of monitoring. However, this evaluation criteria is more subjective and should not be weighted as much as the objective criteria suggested earlier.

On the other hand, the effectiveness of the prevention mechanism of eKwip can be assessed and compared against the traditional mechanical braces. Patients who have recovered from ACL injuries will choose to wear either a mechanical brace, eKwip, or no braces at all. The reinjury rate across the three groups can then be collected through a survey and the effectiveness of each method can be successfully compared.
