
% Many thanks to Andrew West for writing most of this file
% Main LaTeX file for CIS400/401 Project Proposal Specification
%
% Once built and in PDF form this document outlines the format of a
% project proposal. However, in raw (.tex) form, we also try to
% comment on some basic LaTeX technique. This is not intended to be a
% LaTeX tutorial, instead just (1) a use-case thereof, and (2) a
% template for your own writing.

% Ordinarily we'd begin by specifying some broad document properties
% like font-size, page-size, margins, etc. -- We have done this (and
% much more) for you by creating a 'style file', which the
% 'documentclass' command references.
\documentclass{sig-alternate}
 
% These 'usepackage' commands are a way of importing additional LaTeX
% styles and formattings that aren't part of the 'standard library'
\usepackage{mdwlist}
\usepackage{url}

\begin{document} 
\nocite{*}
% We setup the parameters to our title header before 'making' it. Note
% that your proposals should have actual titles, not the generic one
% we have here.
\title{CIS400/401 Project Proposal Specification - eKwip}
\subtitle{Dept. of CIS - Senior Design 2013-2014\thanks{Advisor: LeAnn Dourte (dourte@seas.upenn.edu).}}
\numberofauthors{4}
\author{
\alignauthor Alex Yau \\ \email{ayau@sas.upenn.edu} \\ Univ. of Pennsylvania \\ Philadelphia, PA
\alignauthor Jacy Clare \\ \email{jclare@sas.upenn.edu} \\ Univ. of Pennsylvania \\ Philadelphia, PA
\and 
\alignauthor Jeffrey Shih \\ \email{jeshih@sas.upenn.edu} \\ Univ. of Pennsylvania \\ Philadelphia, PA
\alignauthor Kunal Mahajan \\ \email{mkunal@seas.upenn.edu} \\ Univ. of Pennsylvania \\ Philadelphia, PA}
\date{\today}
\maketitle

% Next we write out our abstract -- generally a two paragraph maximum,
% executive summary of the motivation and contributions of the work.
\begin{abstract}

  \textit{This senior design project focuses on creating a electronic knee wrap for athletes to wear in order to determine their risk of suffering an Anterior Cruciate Ligament (ACL) injury. ACL injuries are among the most common injuries in sports. While there has been a lot of research into some of the possible factors that lead to ACL tears, the project will focus on a few factors and use those to come up with an algorithm to calculate an athlete's risk factor for an ACL injury.}

  \textit{The aim is to come up with a project that will be able to replace the common techniques used by therapists and physicians to determine ACL risk and recovery. In addition, the project will also be multi-purpose, allowing both doctors to predict injuries and athletes prevent injuries on the field.}

\end{abstract}

% Then we proceed into the body of the report itself. The effect of
% the 'section' command is obvious, but also notice 'label'. Its good
% practice to label every (sub)-section, graph, equation etc. -- this
% gives us a way to dynamically reference it later in the text via the
% 'ref' command.
\section{Introduction}
\label{sec:intro}
Anterior cruciate ligament (ACL) ruptures are one of the most common injuries in sport and very expensive to treat. There are approximately 175, 000 primary ACL reconstruction surgeries performed annually in the USA with an estimated cost of over \$2 billion US \cite{yu2007mechanisms}. This is followed by a lengthy rehabilitation period. Even then, over a quarter of patients reinjure their ACL \cite{stevenson1998gender}. An ACL injury can have a immense impact on an athelete's career and quality of life. Because of this, preventing ACL injuries is exceptionally valuable and important.

This project proposes the use of an electronic knee wrap (eKwip), to predict and prevent ACL injuries in both healthy athletes and injured patients. By monitoring the angles, orientation and movements of the knee of the wearer, eKwip is able to predict when an injury is imminent. This will then trigger the prevention mechanism to cushion knee and prevent the knee from bending at angles that may cause an ACL rupture. To encourage adoption of such a knee wrap, eKwip is designed to be unobtrusive and flexible, unlike the current mechanical braces on the market.

eKwip allows the wearer to monitor his or her knee performance and risk of injury throughout the day to decrease the chance of future injury. It also allows physical therapist to easily observe and assess the performance of injured patients remotely. With eKwip, this project aims to reduce the rate of ACL injuries in both healthy athletes and injured patients.

% The header of this document might have been a little intimidatating
% to beginners. Notice once you are in the body of the document,
% however, LaTeX commands are minimal and 'normal text' is frequent.
\section{Related Work}
\label{sec:related_work}
Related work for this project is divided among three fields: algorithms for ACL injury detection (Section 2.1), knee brace effectiveness/performance hindrance (Section 2.2), and Smart Materials (Section 2.3). ACL injury detection covers previous research in minimizing the testing required to detect whether an athlete is at risk of suffering an ACL injury. Knee brace effectiveness/performance hindrance deals mainly with studies of the extent to which a Functional Knee Brace and a Prophylactic Knee Brace play in the ability for an athlete to perform on the field. Finally, Smart Materials concern the reach for this project and the possibility of actually preventing the ACL injuries from occurring on the field.

\subsection{ACL Injury Detection} Prior research in ACL Injury Detection have shown that techniques exist that accurately capture and analyze various measures relating to the knee to determine the probability of ACL injuries \cite{smedicine}. Using such metrics as knee valgus motion, knee flexion rotation of motion, and body mass, researchers were able to come up with a way to determine knee abduction moments (KAM), which are used to identify whether or not an athlete is at high risk for an ACL injury, with a sensitivity of 77\% and specificity of 71\% \cite{smedicine}\cite{Bahr01062005}. 

\subsection{Knee Brace Effectiveness/ Performance Hindrance} Prior research in using Functional or Prophylactic knee braces out in the field useful but results on the potential hindrance of using these kinds of braces were inconclusive \cite{Myer01042011}. Knee braces, especially Functional Knee Braces (FKB), which are more mechanical in nature and thus more obtrusive, are shown to provide “20-30\% greater knee ligament protection”. This suggests that FKBs have an impact in reducing the severity of knee injuries. More testing needs to be done to see whether or not FKBs will actually hinder the performance of an athlete. Another important factor of the effectiveness of knee braces is in rehabilitation, where a combination of exercises and brace use can speed up recovery \cite{hewett2010acl} .

\subsection{Smart Materials} Prior research in the use of Electro-rheological fluids (ERF) show that these material reacts quickly when presented with an electric field, have a high yield stress, and are very lightweight and easily molded \cite{smaterials}. ERF can quickly change viscosities, which make it an excellent material to use in the project. Due to the very nature of ERF’s fast response and simple interface, using it in a light, functional knee brace would allow lead to an easy implementation of the preventative nature of the project.

\section{Project Proposal}
\label{sec:project_proposal}
This project focuses on the research and development of eKwip, an electronic knee wrap for injury prediction and prevention, for both healthy athletes as well as patients recovered from an ACL injury. Unlike traditional mechanical knee braces, eKwip is unobtrusive and flexible. It is able to adapt to and learn the pattern of movements of its wearer, closely monitor the performance of his or her knee, predict when an injury is imminent and reduce the impact of the injury. Through eKwip, the wearer or physical therapist is able to monitor the performance of the knee and assess the risk of injury in the future through a simple user interface. This can allow both healthy athletes to reduce the risk of injury, as well as allow injured patients to monitor their risk of reinjury over time. eKwip also provides an easy way for physical therapist to observe their patients and judge whether they are fit to return to sports.


\subsection{Anticipated Approach}
\label{subsec:approach}
\input{sections/approach.tex}

\subsection{Technical Challenges}
\label{subsec:tech_challenges}
There are several technical and non-technical challenges expected from this project. The first major challenge includes determining the variables to be measured from the knee in order to effectively predict the risk of injury. This includes a lot of experimenting, researching and consulting experts in field to correctly identify the features that best indicates an imminent ACL injury. Since there is a trade off between measuring and collecting as much information as possible, and the storage cost and processing power of the microcontroller, only the critical variables should be identified and used in the prediction process. Given the set of necessary measurements, such as relative angle, speed and orientation of knee, the placement of the sensors need to be designed to be as unobtrusive to the wearer as possible while still able to collect precise and accurate measurements from the knee. With this set of measurements defined, the knee can be effectively modelled. Currently, aside from the knee valgus motion, knee flexion range of motion, and body mass, there may be more measurable factors directly contributing to ACL injury that have yet to be determined, and thus will be a huge challenge for this project \cite{Myer01042011}.

The second major challenge involves determining the risk of injury for each individual. Since the different wearers are likely going to have different range of tolerable movements based on their fitness, gender, age and training, it is necessary for the wrap to be adaptive and flexible. Thus, the wrap needs to be able to use the collected data to make adjustments to its prediction algorithm through reinforcement learning. The challenge, however, is that the knee wrap will mostly be exposed to healthy movements in which an injury is not about to occur. It lacks the data for when an injury is occurring and therefore negative feedback is not possible. Therefore, the learning algorithm needs to be designed to be able to learn the range of movements of the wearer with only the limited dataset available.

Another major challenge is to come up with a risk index given the collection of data. Since the goal of the risk index is for it to eventually become a universal way to estimate the risk of ACL injury, it is necessary for the index to be as accurate as possible, and customized to the physique of the wearer. The construction of such an index is very complicated as it needs to take into account of the physique of the individual wearing the wrap, combined with the index currently used in rehabilitation, and assessments used to judge whether athletes are ready to return to sports after an injury \cite{Sinkjar1991209}. Although this is an ambitious task, and haven’t been accomplished before, coming up with a universal risk index to assess the possibility of ACL injury and reinjury is very useful for monitoring the performance of athletes or injured patients.

A major technical challenge to this project is to be able to collect measurements at a high enough frequency to predict and prevent the ACL injury in real time. Depending on the frequency of measurements needed to allow enough time for the microcontroller to calculate and react, some parts of the code may have to be written in a low level language, such as assembly, to decrease the processing time. The first challenge here is to identify these bottlenecks in the code and the subsequent challenge is to optimize those sections to run as fast as possible without sacrificing precision and accuracy of the injury prediction. This can allow eKwip to detect and predict ACL injuries in real time and react as fast as tens of milliseconds.

The project also presents some design challenges. The eKwip wrap should be designed as comfortable and unobtrusive as possible to not hinder the movements of athletes and patients until an injury is imminent. The wrap should be light and flexible to encourage athletes to wear it even if they have a healthy knee. This requires careful choice of microcontrollers, sensors and other components in the wrap, as well as the positioning of such components. Ultimately, eKwip should be designed to be flexible to fit athletes and patients of different body types.

Lastly, the project also presents some material and mechanical challenges. In order to prevent the ACL injury, the wrap has to be equipped with a prevention mechanism capable of absorbing huge forces and supporting the weight of the wearer. A possible approach, as mentioned in 3.1.3, is the use of smart materials that is capable of changing viscosity. However, in order to create substantial support for the knee, a large electric field or magnetic field has to be used. Depending on the size of field, this approach may be impossible for a portable wrap like eKwip. This will be explored further if time permits. 

\subsection{Evaluation Criteria}
\label{subsec:eval_criteria}
In order to determine whether eKwip performs as intended, it is necessary to evaluate the knee wrap after the prototype is built and compare its performance with the current knee wraps. Since there are currently no knee wraps on the market that accomplishes all of what eKwip sets out to do, eKwip's different functionalities are to be compared with the different equivalent products and services out there.

To determine the effectiveness and accuracy of the injury prediction and risk assessment part of eKwip, one can compare it to other commonly used knee stability assessments in rehabilitation centers , such as the use of electromyographic measurements from muscles acting on the knee \cite{Sinkjar1991209}. Patients who suffered ACL injury can wear the eKwip knee wrap and their performance on those knee stability assessments can be compared to the risk prediction rating generated by eKwip to see if eKwip correctly identifies the health of the injured knee. If a correlation is found, then eKwip is able to successfully and reliably measure the health of a person’s knee without needing him or her to undergo a set of assessments under supervision. 

Other than performing correlation analysis on eKwip, one can also judge the quality and effectiveness of eKwip through the opinions of experts in the field. Doctors and physical therapists can use eKwip to monitor their patients and determine whether eKwip is easier and as accurate at monitoring patients over the traditional ways of monitoring. However, this evaluation criteria is more subjective and should not be weighted as much as the objective criteria suggested earlier.

On the other hand, the effectiveness of the prevention mechanism of eKwip can be assessed and compared against the traditional mechanical braces. Patients who have recovered from ACL injuries will choose to wear either a mechanical brace, eKwip, or no braces at all. The reinjury rate across the three groups can then be collected through a survey and the effectiveness of each method can be successfully compared.


\section{Research Timeline}
\label{sec:research_timeline}
The project timeline is split into the following sections - already completed, prior to thanksgiving, prior to Christmas, completion and if time permits.

\begin{enumerate}
    \item Already completed:
    \begin{itemize*}
        \item Ordering of materials required for first prototype
        \item Speak with experts in field to find out possible causes of ACL tear
        \item Speak with doctors to find out the benefit of such a product to the field
    \end{itemize*}
    \item Prior-to thanksgiving
    \begin{itemize*}
        \item Prototype built
        \item Preliminary data collected
    \end{itemize*}
    \item Prior-to christmas
    \begin{itemize*}
        \item Phase 1 completed
        \item Tweaking of sensors and measurements on prototype
        \item Allow data to be stored on microcontroller
        \item Capable of displaying data collected in a user friendly interface.
    \end{itemize*}
    \item Completion tasks
    \begin{itemize*}
        \item Phase 2 completed
        \item Speak with experts in field to determine risk index
        \item Conduct more testing to improve accuracy of measurements
        \item Real time prediction of injury
        \item Streaming of data from wrap to server
        \end{itemize*}
    \item If there's time
    \begin{itemize*}
        \item Phase 3 completed
        \item Research material needed for injury prevention mechanism
        \item Integrate material with eKwip
        \item Connect the prevention mechanism to the prediction algorithm
    \end{itemize*}
\end{enumerate}



% We next move onto the bibliography.
\bibliographystyle{plain} % Please do not change the bib-style
\bibliography{sections/references}  % Just the *.BIB filename

\end{document} 

